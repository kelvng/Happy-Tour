\section{TỔNG QUAN VỀ ĐỀ TÀI}


\subsection{Giới thiệu đề tài}


\textTo{ Khi làn sóng đại dịch COVID-19 kết thúc, nhu cầu du lịch sẽ bùng nổ trên toàn thế giới trong đó có Việt Nam. Để đón đầu nhu cầu du lịch ngày càng cao và đảm bảo chất lượng dịch vụ, quản lý hiệu quả và tiếp cận khách hàng dễ dàng hơn, chúng tôi đã triển khai \textbf{hệ thống quản lý và đặt tour du lịch - Happy Tour }. Giúp khách hàng dễ dàng đặt vé trực tuyến và giúp chủ doanh nghiệp dễ dàng quản lý đội ngũ nhân viên và các tour du lịch. Từng bước tin học hóa mọi thông tin dữ liệu và quản lý và thay thế dần các công tác quản lý thủ công của cửa hàng sang trực tuyến. Qua đó đem lại nguồn doanh thu lớn hơn cho doanh nghiệp.}

\subsection{Mục tiêu đề tài}

\textTo{Thiết kế một website du lịch phải đảm bảo tính tiền dùng đối với người dùng và quản trị viên. Nội dùng và cách trình bày website phải rõ ràng mạch lạc và dễ sử dụng. }

\begin{itemize}
\item Vận dụng các kiến thức đã học về phát triển hệ thống thông tin doanh nghiệp

\item Áp dụng các kiến thức đã và đang học về lập trình web để triển khai dự án thực tế
\end{itemize}

\subsection{Giới hạn và Phạm vi đề tài}

\subsubsection{Giới hạn đề tài}

\begin{itemize}
\item Đề tài chỉ được xây dựng trên nền tảng website với các kích thước màn hình chuẩn cho thiết bị máy tính thông dụng chưa áp dụng responsive để phù hợp với các thiết bị di động.
\item Thiết bị sử dụng hệ thống phải được cài đặt NodeJS trước khi sử dụng.
\end{itemize}

\subsubsection{Phạm vi đề tài}

\begin{itemize}
\item Phạm vi không gian: Đề tài được nghiên cứu trong phạm vi lãnh thổ Việt Nam có thể vướng mắt về pháp lí ở các nước ngoài.
\item Phạm vi thời gian:Đề tài được nghiên cứu trong thời gian 3 tuần .
\end{itemize}

\subsection{Nội dung thực hiện}

\subsubsection{Những khó khăn khi đặt vé và quản lý thủ công?}

\begin{itemize}
\item Việc tìm hiểu thông tin tour, lịch trình, giá cả,… mất nhiều thời gian và không chính xác.
\item Khó khăn trong việc tìm kiếm khách hàng mới.
\item Khó thông báo các chương trình khuyến mãi đến khách hàng
\item Mất nhiều thời gian để tổng hợp, duyệt thông tin, lên lịch trình,...
Ví dụ: Trưởng đoàn muốn lấy thông tin về chuyến tham quan ngày mai.
\item ...
\item Khó đáp ứng được sự phát triển ngày càng lớn của cửa hàng trong tương lai

\end{itemize}

\subsubsection{Những thuận lợi khi đặt vé và quản lý trực tuyến?}

\begin{itemize}
\item Việc tra cứu thông tin tour, lịch trình, giá cả,… nhanh gọn mà còn chính xác.
\item Dễ dàng trong việc tìm kiếm khách hàng mới nhờ quảng cáo website trên các nền tảng mạng xã hội.
\item Dễ dàng thông báo các chương trình khuyễn mãi đến khách hàng.
\item  Tổng hợp, duyệt thông tin, lên lịch trình,.. nhanh gọn, chính xác.
\item ....
\item Đáp ứng được sự phát triển ngày càng lớn của cửa hàng trong tương lai
\end{itemize}


\subsubsection{Tại sao nên sử dụng hệ thống quản lý đặt vé du lịch Happy Tour ?}

\begin{itemize}
\item Khách hàng sẽ tiết kiệm thời gian, chi phí trong việc tra cứu thông tin chuyến du lịch mà còn chính xác hơn.
\item Dễ dàng kết nối tư vấn giữa khách hàng và nhân viên.
\item Dễ dàng nhận thông tin chuyến du lịch mới, chương trình khuyến mãi, ... phương thức thanh toán.
\item Dễ dàng nắm bắt thông tin chuyến du lịch đã đặt một cách nhanh gọn và chính xác.
\item ...
\item Hệ thống quản lí nội bộ đi kèm giúp doanh nghiệp dễ dàng quản lý thông tin về tour và nhân viên
\end{itemize}

\subsection{Ý nghĩa thực tiễn}

\textTo{Website được ứng dụng vào thực tế  đáp ứng đầy đủ yêu cầu của công ty. Về phía khách hàng, thuận tiện cho khách hàng.}

\subsection{Phương pháp nghiên cứu}

\textTo{Trong quá trình thực hiện đề tài em có sử dụng một số phương pháp nghiên cứu}

\begin{itemize}
\item Thu thập thông tin thông qua ghi chép, phân tích, báo cáo từ công ty.
\item Sử dụng phương pháp suy luận, tư duy biện chứng để đưa ra nhận xét và kiến nghị.
\item Và một số phương pháp khác.
\end{itemize}

\subsection{Cấu trúc đề tài}

\textTo{Để phù hợp với mục tiêu, phạm vi, đối tượng cũng như nội dung bố cục đề tài như sau:}

\begin{itemize}
\item Chương 1: Giới thiệu tổng quan về đề tài
\item Chương 2: Phân tích thiết kế hệ thống
\item Chương 3: Hiện thực hệ thống
\item Chương 4: Demo
\item Chương 5: Kết luận
\end{itemize}











